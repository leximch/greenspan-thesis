\chapter{Conclusion}

In its most general terms, this thesis aims to connect abstract philosophical thought to concrete material practices. It does so by concentrating on the convergence between, on the one hand, one of the most abstruse areas of philosophy, that is the transcendental analyses of time and, on the other, on the actual or concrete changes in the technology of time-keeping systems (and the socio-cultural and economic transformations that necessarily accompany these shifts). However, despite obvious isomorphies and interlinkages between these two 'spheres' the topics are rendered irreconcilable by a process of reciprocal interiorization that opposes the nature of time (conceived of as an epistemological subjectivity) with the innovations or changes which occur within history (understood as a narrative unity).

Thus to connect the transcendental philosophy of time with the sociotechnics of time-marking processes requires the disorganization of both terms onto a plane of exteriority---or absolute immanence---where change occurs not in time but to time. This involves the reanimation of a systematic philosophy of abstraction which is drawn primarily from the work of Deleuze and Guattari. This systematic abstraction is a materialist philosophy, not in the sense that it is a theory about matter but rather due to its practical or pragmatic orientation which treats concrete material events as the instantiation of abstract machines which are themselves 'philosophical' in nature.

The thesis begins by outlining the discussion of time found in Immanuel Kant's \textit{Critique of Pure Reason}. It seeks to show how, through the discovery of the transcendental, Kant constructs an account of the abstract production of time which is freed from its ties to variation and movement. Time thus becomes a formal structure which is constant and exists in an exterior relation to the changes which occur inside it.

In the second chapter this notion of transcendental time is linked to the temporal transformations which have occurred under capitalism by way of the invention of the mechanical clock. It does this by mapping the Kantian notion of formal time onto the quantitative and homogenous time of the clock, a time which is characterized by its autonomy from the calendar, its standardization throughout the globe and its rigorous identification with money.

Yet, despite obvious parallels, the socio-history of capitalist time and Kant's critical philosophy remain separate and opposed. This separation corresponds to the fundamental distinction---which both sides insist upon---between historical change and the epistemological structure of time, an opposition which results in each side seeking to envelop the other by way of its own superior unity (which on one side is constituted by the transcendental unity of the subject and on the other the unity of the historical process).\footnote{For Marx this unity culminates in the universality of proletarian class consciousness.} Linking them together, thus always seems to occur at the expense of one side or the other, (whether this be through the subordination of the nature of time to historical variation or vice versa).

Nonetheless, both Kantian thought and the production of capitalist time, involve---perhaps despite themselves---the dissolution of this distinction, or relation of mutual transcendence. For both are aspects of a singular event, or revolution, which occurs not in time but to time. Thus, though they appear as revolutions in history, they are not in themselves historical. For in altering time itself, both have accessed an abstract realm which conditions experience, and which impacts the smooth succession of history only from the outside. Thus, the Kantian and capitalist revolution introduce a mutation or radical discontinuity in both the thought and the material practices of time.

In the classical western tradition the only thing capable of changing the very nature of time, was the exterior and transcendent power of the eternal. Since it is eternity which is ultimately responsible for the production of time, changes in the nature of time---if they were ever possible---could only be transcendentally produced. Restricted to initiations and terminations, these changes were a matter of faith not reason. For they occurred only as the apocalyptic or miraculous events which accompany divine revelation.\footnote{In Christianity the intrusion of the eternal in history is ultimately realized through the incarnation of Jesus, or the idea of the Word made flesh.}

Both critique and capitalism, however, are intrinsically opposed to transcendent impositions. Though they push aside any explicit engagement with the question of eternity, they nevertheless reject this classical conception of the exteriority. Operating with consistent circuits of abstract and effective production they tend to the construction of immanent systems which obsolesce structures of faith, essentialized authority, and arbitrary and supplementary dimensions.\footnote{Capitalism's tendency to monetarize power differentiates it from other social systems, in which power is based primarily on coded and territorial structures of organization. According to Deleuze and Guattari capitalism is defined by these processes of decoding and deterritorialization, and it is this which is responsible for its great affinity with immanence.} Dedicated in their principles to an immanence of criteria, both imply---however implicitly---the necessity for an immanent production of time to replace the faith in transcendent creation. 

However, due to the stratified distinction\footnote{As has already been noted, stratification acts as double pincer, operates through double articulation, and constitutes the world through binary distinction (See \textit{A Thousand Plateaus}, esp p. 40).} in both, between, on the one side, an idealist structure of time (whether that be the logic of historical development or the universality of \textit{a priori} synthesis) and on the other, the variations which occur in time, neither Kant nor the social history of capitalist time can provide the conceptual immanence which their principles require. In retaining the notion that the interiority of temporal variation is constituted by the intrinsic unity of a higher and more primary structure, they dismantle the traditional faith in eternity but nevertheless leave a quasi transcendence in place. Incapable of conceiving of a variation that is flat with the construction of time, and that abstract synthesis are consistent with the multiplicity and becomings of material innovation, they deny the very possibility of time mutation. Blind to the implications of their own respective revolutions they thus conceal the intensive plane on which time is immanently produced.

In order to uncover this plane, the thesis turns to the writings of Deleuze and Guattari whose work \textit{Capitalism and Schizophrenia} calls for a materialist revolution in the name of transcendental thought. Pushing the Kantian system further in the direction of immanent critique, Deleuze and Guattari manage to dismantle the distinction between conceptual abstraction and material innovations. Though they retain the exteriority of transcendental synthesis they cease to locate these syntheses inside the mind of the knowing subject and see them instead as the operations of abstract machines. Deleuze and Guattari thus discover a plane of consistency on which the nature of time is flat with the variations and mutations that are intrinsic to its own production. 

A philosophy based on the immanent production of time requires, as we
have seen, not only a reformulation of the nature of time but also of the relation between time and eternity. To reach the plane of consistency the implicit faith in transcendent exteriority must be replaced with the participation of an immanent outside. Deleuze and Guattari thus substitute the division between time and eternity with the difference between two modes of temporality; the extensive time of Chronos and the intensive time of Aeon. While Chronos corresponds to the stratified nature of time (with its division between structure and change), the temporality of Aeon is constituted by becomings, intensive variations, machinic multiplicities and singular events which do not differentiate between the abstract production of time and material innovation (which is generally considered to occur in time). 

Composed on the plane of consistency or immanence, Aeon does not transcend, interrupt or break into time in the same way that eternity does. Rather, as we have seen, it constitutes the virtual field upon which Chronos is continuously being constructed. Functioning in this way, Aeon is not to be understood as an eternal or abstract generality. Instead the virtuality or abstraction of Aeon can only be accessed through the singular events out of which it is composed. The final chapter of this thesis thus explores the concept of Aeon by focusing on one such singular event, the dawn of the third millennium, a fundamental juncture in the passage of time and in time-keepings socio-technical apparatus. This event has come to be known by the sign Y2K.

Post millennium cynicism is such that it seems absurd to even mention Y2K, never mind speak of it as an event of fundamental philosophical importance. For it is now generally agreed that, though it was hyped to apocalyptic proportions, Y2K was---if anything---non-event. Though glitches were reported---even in such crucial areas as stock exchanges, transportation networks, emergency services, and credit card companies---each of these was easily dealt with on an individual basis and did not seem to add up to anything significant (certainly it was nothing like the global catastrophe that was predicted). As an event, Y2K was so diffuse, so quiet, so inconsequential that its very existence has been retrospectively called into question. After all, despite months and even years of anticipation January 1\textsuperscript{st} 2000 seems to have been just another day.

The real nature of Y2K still remains a puzzle. No one is sure whether the hundreds of billions of dollars spent were wasted or whether they were crucial in the prevention of a catastrophe. While some maintain that the risk was wildly exaggerated, the few 'glitches' that did occur are sufficient to give evidence that there was indeed a problem. Yet, the fact that countries which appeared to do so little to fight the bug (i.e. Russia and China) encountered no more disturbances than countries like United States and Britain which reacted early and poured huge resources into ensuring 'millennium compliance' makes the conclusion that the problem was fixed highly improbable. The alternative, however, that Y2K was a vast conspiracy by the computer industry, is even more preposterous.

The confusion which - even now -surrounds Y2K is a result of the fact that---though entirely real---it was an event that never seemed to actualize. Y2K was, and always will be, a virtual catastrophe, a pure potentiality, a non-event. The final chapter of this thesis argues that the virtual nature of Y2K---which allowed it to be entirely effective (as a potentiality) and yet never empirically manifest---suggests that it cannot be understood through the successive temporality of Chronos. Rather, Y2K is a sign - which operates as both a name and a date---for an event composed on the intensive plane of Aeon. This, as we have seen is evidenced by its efficient nonsignifying (numerical) semiotic, its resilient virtuality, its disorganization of linear succession, and its dissolution of such stratified distinctions as content and expression, quantity and quality, constant and variable, and technics and culture.

It is as an Aeonic event that Y2K makes the connection between the transcendental philosophy of time and the socio-economics of capitalist timekeeping practices. Flat with an exterior plane of machinic abstraction, it dissolves the distinction between time and the materiality of time-keeping systems. For this reason, Y2K has been used as an exemplary event in addressing the central problematic of this thesis.

There is no question that despite the pervasive sense of anticlimax, Y2K was a crucial event in the history of capitalist time. For though it acted only as a potentiality, it had concrete material consequences whose effects can be measured in billions of dollars. As a technological 'glitch' in cyberspace time, Y2K mobilized the global economy in an unprecedented fashion. Operating within the context of an ever-increasing convergence of time and money, it turned the date that marked the end of the second millennium into the most expensive accident the world has ever experienced.

What makes Y2K crucial to the philosophy of time, however, is that it is also a near perfect example of systematic abstraction. The word abstraction means to extract, remove or withdraw from any particular concrete instantiation. It thus involves a process through which a particular dimension or aspect of any given context is subtracted and made autonomous. To quote from Deleuze: 

\begin{quote}
    The way people talk about abstraction is absolutely amazing, they have absolutely no idea what it is. Philosophy has a kind of technique or terminology like mathematics. Generally the word abstract is used for things in which there is no abstraction. The problem of abstraction is how can I make two things out of what only exists as one in my representations. It's not difficult to make a thing into two when I have two representations, but when I say the back of the piece of paper, I am not abstracting at all since the back is given to me in a representation which itself exists. When I say a length without thickness, there I am abstracting because I am separating two things which are necessarily given in each other in my representation (TLK, 5).
\end{quote}

Y2K functions as a model of abstraction in that it subtracts the first two
digits from the date. Through this subtraction, it serves to extract the decade, making it autonomous from the interiority of the century. It thus separates out two things from what--until then---had appeared only as one (rupturing the apparent unity of historical time). Y2K---like every abstraction---is a schism. By fracturing the semiotic expression of dated time, it abstracts a scale of time from the history within which it is previously embedded (the year 00, for instance, could belong to any century whatsoever). The anticlimactic character of Y2K only confirms its nature as an abstract event, through which time has escaped from the concrete interiority of history.

As this thesis has argued the tendency of capitalist chronometrics---from its inception--has been affined with the transcendental in its trend towards the ambiguous 'liberation' of time from the measurement of change (thus establishing the autonomy of the clock in relation to the calendar). In this respect Y2K figures as a culmination, although a paradoxical one, since it machinically extracts an abstract temporality from dates themselves. With Y2K, dates cease to function solely as the expressions of celestial cyclicity and implicit historical unity. Instead they are activated as numerical indices for pure---or Aeonic---events, marking absolute historical schisms which correspond to thresholds of
innovation in the abstract production of time.

Lecturing on the \textit{Critique of Pure Reason}, Deleuze maintains that fundamental to Kant's discovery of the immanent plane of the transcendental is that "true lived experience is an absolutely abstract thing\dots once you have reached lived experience, you reach the most fully living core of the abstract".
"Nobody," Deleuze continues, "has ever lived anything but the abstract" (SLK, 5).

This thesis has focused on transcendental philosophy in order to investigate the abstract nature of time. What it has discovered, is that this most obscure and seemingly distant topic is encountered in the technology or socioeconomic practices of contemporary life. For the nature of time is not some eternal given that has descended from above, but is rather a process that is itself continuously under production, though not---as Kant believed---in the interiority of thought, but rather on the exteriority of an unconscious, immanent, and material plane of machinic transformation. 
\chapter{Philosophy's Copernican Revolution: Time in The \textit{Critique of Pure Reason}}

\chapterprecishere{The history of Immanuel Kant's life is difficult to portray, for he had neither life nor history ... I do not believe that the great clock of the cathedral performed in a more passionless and methodical manner its daily routine than did its townsman, Immanuel Kant. Rising in the morning, coffee drinking, writing, reading lectures, dining, walking, everything had its appointed time, and the neighbours knew that it was exactly half-past three-o-clock when Immanuel Kant stepped forth from his house in his grey, tight fitting coat, with his Spanish cane in his hand, and betook himself to the little linden avenue called after him to this day the `Philosophers walk. "... What a strange contrast did this man's outward life present to his destructive, world annihilating thought! In sooth, had the citizens of Konigsberg had the least presentiment of the full significance of his ideas, they would have felt a far more awful dread at the presence of this man than at the sight of the executioner, who can but kill the body. But the worthy folks saw in him nothing more than a Professor of Philosophy, and as he passed in his customary hour, they greeted him in a friendly manner and set their watches by him\par\raggedleft--- \textup{Heinrich Heine}, Religion and Philosophy in Germany, \textup{p. 109.}}

\newpage
% TODO: Fix quote/epigraph format.
\chapterprecishere{\par\raggedleft\textbf{\textup{49. Ko. Revolution.}} \begin{flushleft}\textbf{\textup{[Molting]}}\end{flushleft}\ Fire in the water. The image of REVOLUTION. Thus the superior man sets the calendar in order and makes the seasons clear.\par\raggedleft---I Ching: The Book of Changes, \textup{p. 190.}}

\section{Introduction: The Discovery of Transcendental Time}

In 1781 Immanuel Kant published the first edition of the \textit{Critique of Pure Reason}. The history of philosophy registers the date as the moment of Kant's "Copernican revolution, " a moment in which Kant is said to have accomplished in the realm of thought what Copernicus had accomplished, over two centuries earlier, in the realm of astronomy. Traditionally, the story of Kant's revolution emphasizes epistemology and concentrates on the role of the human intellect in constituting the external world. It is here, we are told, that one should locate the dramatic shift that is at the core of Kantian philosophy.\footnote{ In the following section this position will be illustrated through the writings of Heinrich
Heine. Though Heine poetic and lyrical language is unique amongst the
commentators on Kant, his views are not at all unconventional. In the '"Past Masters" text on Kant, for example, Roger Scruton, writes that the essence of the Copernican revolution is that 'self consciousness requires that the world must appear to conform to the categories' (Scruton, Roger. \textit{Kant}. Oxford: Oxford University Press, 1982, p. 28). The Encyclopedia of Philosophy concurs that this epistemological view is the key to understanding Kantian thought. 'Kant's principle task in the \textit{Critique of Pure Reason} was to determine the cognitive powers of reason, to find out what it could and could not achieve in the way of knowledge" ("Kant, Immanuel" in the \textit{Encyclopedia of Philosophy Volume 3}. Edited by Paul Edwards. New York: Macmillan Publishing, 1967, p. 305).} What this chapter will argue, however, is that this emphasis, on the role of the intellect, mistakes what is truly revolutionary in critical thought. The argument, which draws on Deleuze's reading of Kant,\footnote{This is comprised both of Deleuze's book on Kant and a series of lectures on Kant that have been reprinted on the Internet (see bibliography).} consists of two main points. First, that the transformation of the human subject is merely a consequence of Kant's more fundamental innovation, the discovery of the realm of the transcendental, and second, that to appreciate the truly revolutionary nature of this discovery we must turn our attention away from the enlightened subject of reason and focus instead on the occulted nature of time.

The most common approach to Kant is to read him as an epistemologist, a philosopher who is predominantly concerned with how knowledge can be justified. "According to this account \textit{The Critique of Pure Reason} centers around the question: "How are \textit{a priori} synthetic judgements possible?" (CPR, 59). To answer this question Kant must begin by defining what he means by the synthetic \textit{a priori}. 

"In the order of time," writes Kant in the preface to the first Critique, "we have no knowledge antecedent to experience, and with experience all our knowledge begins. But though all our knowledge begins with experience it does not follow that it all arises out of experience" (CPR, 41). With this distinction--between knowledge that is based in experience and knowledge that is independent of it--Kant separates the \textit{a priori} from the \textit{a posteriori}. This difference is absolute and rigorously determined. Empirical knowledge which has its sources \textit{a posteriori}, that is, in experience" (CPR-43) gives rise to judgments which are particular and contingent (the sun rose today) while \textit{a priori} knowledge, on the other hand, gives rise to judgments which are universal and necessary (today succeeded yesterday). Since knowledge gained from experience is always particular and contingent it can never be the basis for judgments which are universal and necessary. Thus, Kant writes, "necessity and strict universality are the sure criteria of \textit{a priori} knowledge" (CPR, 44). This difference between the \textit{a priori} and the \textit{a posteriori} is the first basic division which allows Kant to demarcate the singular zone of knowledge that he is concerned with in the first Critique.

Yet this distinction, between knowledge that is based in experience and knowledge that is independent of it, is not in itself sufficient for understanding the central problematic of Kantian thought. In order to discover the transcendental, yet another distinction was required. The two sides of the table had to be split in half. To accomplish this, Kant drew another line on a different axes, cutting across both the \textit{a priori} and the \textit{a posteriori}.\footnote{See Table 1 in the appendix.} This line corresponds to the difference between knowledge that is analytic and knowledge that is synthetic. 

Analytic \textit{a priori} cover logical truths. By drawing out "something that is (covertly) contained within a concept" (CPR, 48) they analyze, elucidate, or explicate what is already implicitly known. Kant's example is as follows: "If I say, for instance 'All bodies are extended' this is an analytic judgment. For do not require to go beyond the concept which I connect with 'body' in order to find extension as bound up with it" (CPR, 48).

Analytic knowledge is restricted to the domain of the a priori. It does not arise empirically. For "judgments of experience, " writes Kant, "are one and all synthetic" (CPR, 49) By the term 'synthetic' Kant is referring to knowledge that 'goes beyond the concept'. To make the judgment 'this body is heavy', for example, is to connect concepts 'synthetically'. For since not all bodies are heavy, one cannot arrive at the concept heavy from an analyses of the concept 'body'.\footnote{On page 49 of the first critique Kant uses this example to explain synthetic judgements which are made from experience.}


% Fix Math handling.
Transcendental philosophy concerns itself with the final box in the table--the one that contains the 'synthetic \textit{a priori}'. The puzzle which it sets for itself is how synthetic knowledge can be produced independently of experience. The prime examples of synthetic a priori knowledge are found within the realm of mathematics. As the commentator Alfredo Ferrarin writes, "If what the Critique shows is the possibility of synthetic \textit{a priori} judgements, it is mathematics that takes advantage of this ampliative principle with greatest confidence and success" (KECI, 147.) '7+5=12', to stick with the example that is repeated throughout the secondary literature on Kant,\footnote{Kant uses this example himself both in the first critique and in the \textit{Prolegomena to Future Metaphysics}.} is a synthetic \textit{a priori} judgment. It is \textit{a priori} since it, like all mathematical propositions, is both universal and necessary. It is synthetic since neither the number 7 nor the number 5 has contained within it the number 12. "The concept of 12," writes Kant, "is by no means already thought in merely thinking the union of 7 and 5; and I may analyze my concept of such a possible sum as long as I please, still I shall never find the 12 in it" (CPR, 53).

Though philosophers had long been concerned with such a priori truths as are found in mathematics it was Kant who first recognized them as being synthetic.\footnote{As Deleuze writes, "analytic a priori judgment that meant something, synthetic a posteriori judgment that meant something, but synthetic a priori judgment, that's truly a monster" (KST, 9).} Thus transcendental philosophy, even when confined to epistemology, is far from being a mere exercise in the catalogue of knowledge. For in questioning the tendency to divide everything between the \textit{a priori} analytic judgment and the \textit{a posteriori} synthetic judgment Kant "exposed the insufficiency of certain philosophical categories and conceptual frames" (KST, 9). Focusing his attention on the synthetic \textit{a priori} he exploded old distinctions and in so doing discovered a powerful new machine. 

According to Kant, synthesis is "a blind but indispensable function of the soul". Though "we are scarcely ever conscious" of its power, without it, he insists, "we should have no knowledge whatsoever" (CPR 112). In focusing on the connections and constructions of this hidden realm, the \textit{Critique of Pure Reason} develops a 'synthesized way of handling philosophy'\footnote{This phrase comes from a question raised by Deleuze in an introductory lecture on Kant. "Why," Deleuze asks, "wouldn't there also be a synthesized or electronic way of handling philosophy" (SLK, 1).} which is not based on analyses of that which is already given but on an "extension of our previously possessed concepts" (CPR, 47). It is, writes Kant "a genuinely new addition to all previous knowledge?" (CPR, 51).

It is this 'new addition' that accounts for the shift in the subject's position which occurs in Kantian thought. What is important to recognize here, is that this shift is a result, a corollary, of the more fundamental discovery of this abstract and productive realm of knowledge. Before Kant, the subject was found buried, submerged underground, chained in the darkness of Plato's cave. According to this traditional vision, the subject was trapped in the body, forced to access the world through the unreliability of the perceptual apparatus. An unfortunate fool, blinded by ignorance, duped into mistaking shadows for reality, the subject could not help but deform the world, mutating it into the falsity of illusion. Philosophy's striving consisted in its promise to provide the escape route. Operating with a truth that depended on a "harmony between the subject and the world," (KST, 5) philosophy struggled to cut the chains, to correct the inherent deformity, to free the prisoner from the world of shadows and illusions. 

Through his discovery of the transcendental, Kant replaces harmony with circuitry. The subject, no longer deceived and defective, becomes productive and constitutive. Having given up the impossible attempt at conforming to the objects of the world, "the rational being discovers he has new power" (KCP, 14). After Kant the objects of the world must conform to us. "The first thing the Copernican Revolution teaches us," writes Deleuze, "is that it is we who are giving the orders" (KCP, 14). The prisoner has become a legislator. 

As previously noted, it is this shift in the subject's position which is traditionally taken to be at the core of Kant's revolutionary thought. In a dramatic passage, the poet Heinrich Heine, describes this, philosophy's Copernican revolution, as follows:

\begin{quote}
    Formerly, when men conceived the world as standing still, and the sun as revolving round it, astronomical calculations failed to agree accurately. But when Copernicus made the sun stand still and the earth revolve round it, behold! Everything accorded admirably. So formerly reason, like the sun moved round the universe of phenomena and sought to throw light upon it. But Kant, bade reason, the sun stand still, and the universe of phenomena now turns round, and is illuminated the moment it comes within the region of the intellectual orb (RPG, 114).
\end{quote}

Yet, while Heine has captured the drama of Kant's discovery, his account reveals a certain problem. For if the stress is on human reason, Kant's allusions to Copernicus are somewhat puzzling. Before Copernicus, Heine reminds us, the earth stood as the central pivot or axis around which everything else revolved. Modem astronomy, which is based on the Copernican system, removed the earth from this central position, making it equal to any other planet. The Copernican revolution thus derailed us from our privileged status in relation to phenomena. Kant, on the other hand, is said to have done the exact opposite. Whereas Copernicus displaced us from the center of the universe, Kant put us there. Why, then, does Kant speak of his philosophy as Copernican? For it would seem that the emphasis on the human intellect is not a sufficient explanation. Perhaps, if we look more closely, we might find some other reason for this seemingly confused analogy. 

In the early years of the 16th century, Nicolaus Copernicus, a polish astronomer, attained immortal fame by overthrowing the Ptolemaic universe of the ancient world. Frustrated with the impossibility of achieving accurate measurements of astronomical movements, Copernicus began to question Ptolemy's geocentric vision. Instead of assuming that the stars revolved around a still earth, Copernicus thought, to quote Kant, "whether he might have better success if he made the spectator to revolve and the stars to remain at rest" (CPR, 23). Copernicus posited a heliocentric world in which the stars no longer measured time. He explained the day by the earth's rotation on its own axes and the year by its annual cycle around the sun. In the Copernican system, then, it is the movement of the earth which marks out the temporality of the astronomical calendar. 

The \textit{Critique of Pure Reason}, writes Kant, proceeds "precisely on the lines of Copernicus' primary hypothesis" (CPR, 23). Inspired by the astronomers' method, Kant attempted an analogous experiment in philosophy. Frustrated by the inherent instability of metaphysics, critical thought seeks to attain more solid foundations by focusing not on the authority of experience but on the conditions which make experience possible. 

It is well known that Copernicus' discovery met with fierce resistance, both from natural philosophy and from the church. For these two institutions were allied in their commitment to maintaining the authority of Aristotle, who had insisted that the earth stood still. This resistance was heightened by the fact that despite Copernicus' findings the world still appeared to conform to Ptolemy's ancient vision. Copernicus was thus responsible for a strange and mysterious revolution in which nothing seemed to change but through which everything has been transformed. It is in this way, as we will see, that Kant is a true Copernican. For the Copernican revolution, whether in astronomy or philosophy, changes nothing at the level of experience. Our perceptions and even the way we talk about those perceptions have not altered. Phenomena remain the same. The sun still appears to revolve. The earth still appears to stand still. External bodies still appear to be in motion. We still say that the sun rises and sets. The difference is, and this is the revolution, that now everybody knows it is only a manner of speaking.

In a series of lectures on the \textit{Critique of Pure Reason}, Gilles Deleuze maps out a singular and original account of Kant's Copernican revolution which is based neither on epistemology nor on a change in the position of the intellect but on a shift in the nature of appearance itself. 

According to Deleuze, the classical tradition structured the world around a basic opposition. "The whole of classical philosophy from Plato onwards," he writes, "seemed to develop itself within the, frame of a duality between sensible appearances and intelligible essences" (KST, 4). Thus, previous to Kant the world was divided between, on the one hand, the degraded realm of sensation which was based on bodily knowledge and experience, and on the other hand, the realm of ideas, pure forms or essences which were transcendent and therefore untainted by the blemishes of sensation. 

For Plato \textit{a priori} knowledge was the proof of transcendence. His dialogues insist that the very fact that there is knowledge independent of experience shows that reason remembers a time when it was unfettered by the body's cage and was free to gaze upon the pure essence of things.\footnote{One of the most famous examples of the Platonic view of the a priori occurs in the \textit{Meno} where Socrates infers the transcendence of the Forms through a slave's knowledge of geometry. See Plato, "Meno" in \textit{Plato--The Collected Dialogues}. edited by Edith Hamilton and Huntington Cairns. Princeton New Jersey: Princeton University Press, 1961 (especially pages 363-374).} The philosophical distinction between \textit{a posteriori} and \textit{a priori} knowledge was thus, for Plato, evidence of the fact that our capture in the illusory realm of phenomena could be opposed to an exteriority characterized by the transcendent truth of the idea.

The \textit{Critique of Pure Reason} overturns the classical tradition by developing a philosophy that is no longer grounded in this basic opposition. "For the disjunctive couple appearance/essence," writes Deleuze, "Kant will substitute the conjunctive couple what appears/conditions of appearance. Everything is new in this" (KST 4).

With Kant, then, phenomena cease to be trapped by the ancient duality. "It's like a bolt of lightening" (KST, 4). The world of appearances vanish. What is left instead, according to Deleuze, is the apparition. "The apparition, is what appears in so far as it appears. Full stop. I don't ask myself if there is something behind, I don't ask myself if it is false or not false. The apparition is not at all captured in the oppositional couple, in the binary distinction where we find a appearances distinct from essences" (KST, 4). 

No longer bound by the fundamental distinction of classical thought, Kant transforms the meaning and implications of \textit{a priori} knowledge. "In the case of the \textit{a priori}, " writes Deleuze, "Kant "borrows a word but he completely renews its sense. " (KST, 1) For unlike Plato, in Kant the \textit{a priori}, as we will see, is associated with the immanence of abstraction and not the transcendence of the eternal forms. To quote from Deleuze: "Kant is the one who discovers the prodigious domain of the transcendental. He is the analogue of the great explorer - not of another world, but of the upper and lower reaches of this one" (DR, 135). In opposition to the transcendent ideas and logic of the analytic \textit{a priori}, the synthetic \textit{a priori} constitute a continuous process of production that is both exterior and immanent to our experience of the world. The basic question of transcendental philosophy "how are \textit{a priori} synthetic judgments possible?" can thus be restated as follows: given a certain experience what are the conditions that went in to producing it? Kant's answer, as we will see, shows that that which is exterior--or independent--of experience is not a transcendent world above us but rather an immanent outside. It is this which he calls the transcendental.\footnote{According to Deleuze, "The whole Kantian notion of the transcendental is created in order to refute the classical notion of the transcendent. The transcendental is above all not the transcendent" (SLK, 7). }

As was noted in the introduction, the classical disjunction between essence and appearance corresponds to the distinction in the philosophy of time between, on the one hand, the phenomena of temporality and change, and, on the other, the essence of eternity. Overturning this classical duality between essence and appearance requires not only a transformation in way we approach phenomena but also a fundamental reinvention in the philosophy of time. As we will see in the chapter which follows, the discovery of the domain of the transcendental ultimately rests on this reinvention--or revolution--in the nature of time. Deleuze, recognizing this, writes in an introduction to Kant that "all the creations and novelties that Kantianism will bring to philosophy turn on a certain problem of time and an entirely new conception of time (KST, 1). It is to this new conception of time which this chapter now turns.
\newpage

\section{Transcendental Aesthetic: Time as the Form of Inner Sense}
\vskip 4ex
\chapterprecishere{Until now the task we have given ourselves was to represent space, the
moment has come to think time.\par\raggedleft--- \textup{Gilles Deleuze}, Kant: Synthesis and Time, \textup{p. 1.}}

With its central divisions, parts, sections, chapters, books, sections of chapters, and chapters of books, the structure of the \textit{Critique of Pure Reason} seems more like the work of a ramshackle artificial intelligence than that of a human being.\footnote{De Quincy's text 'The Last Days of Immanuel Kant' gives further evidence of this seemingly preposterous claim. Besides the meticulous order of his daily schedule, Kant never perspired, evoked rigorous numerological arrangements for the guests at his dinner table, and, at his deathbed, when all human faculties had left him, was still able to speak at length on any problem in history, philosophy or mathematics. See: De Quincy, Thomas. 'The Last Days of Immanuel Kant' in \textit{The Collected Writings of Thomas De Quincy} Vol. IV. London: A\&C Black, 1897.} The immense scale and complexity of Kant's 'thinking machine' is revealed with one glance at the table of contents. 

The bulk of the text is divided into two main parts, the 'Transcendental Aesthetic' and the 'Transcendental Logic'. This split corresponds to the central distinction in Kantian thought which divides the intuition from the understanding. Intuition deals with the realm of sensation. Receptive and immediate, it is the form in which the diversity of sense material is presented to the mind. Understanding, on the other hand, is defined as the 'spontaneous production of concepts' (CPR, 92). An active mediation, rather than a receptive and immediate presentation, it serves to represent the perceptions that are given to us in intuition in accordance with the categories of reason. Time appears first in the former category. It is defined in the opening section of the \textit{Critique of Pure Reason} as a 'pure intuition,' or the 'form of inner sense' (CPR, 77).

The Transcendental Aesthetic, "the science of a priori sensibility"
(CPR, 66) begins with a strict process of elimination. It is concerned only with what is left after both the concepts of the understanding and the matter of sensibility have been stripped away. What remains are what Kant calls the 'pure intuitions,' or 'the form of appearances'. These are defined as the underlying conditions that constitute our perception of the world. The transcendental media for the reception of sensible content, they constitute the structure and form which the apparition must take. For Kant there are only two such forms, space and time. 

Kant defines space as "the form of outer sense". It is "the property of our mind" through which "we represent to ourselves objects which are outside us" (CPR, 67). Thus, for Kant, space is the form in which the external world is presented to the senses. That is to say, everything that we sense as external to us we necessarily perceive of as being in space. To quote from
the first Critique: 

\begin{quote}
    In order that certain sensations be referred to something outside me (that is, to something in another region of space from that in which I find myself), and similarly in order that I may be able to represent them as outside and alongside one another, and accordingly as not only different but as in different places, the representation of space must be presupposed (CPR 68).
\end{quote}

Time, on the other hand, is defined as "the form of inner sense, that is of the intuition of ourselves and of our inner state" (CPR, 77). In the 'Transcendental Aesthetic,' then, time provides the underlying structure of all our states of mind. It is what conditions the very experience of thought, including our awareness of outer perceptions and the consciousness we have of ourselves. As Kant writes, "everything which belongs to inner determinations is represented in relation to time" (CPR, 68).

In the following section we will see that in making time the form of inner sense Kant revolutionizes the classical philosophy of time both by liberating it from its dependence on change and by releasing it from an implicit spatial bias. 

According to Deleuze, the first great Kantian reversal in the \textit{Critique of Pure Reason} was to free time from its subordination to movement Taking Hamlet's phrase the 'time is out of joint' and applying it to Kant,\footnote{In his text "On four poetic formulas that might summarize the Kantian philosophy" (found in both \textit{Kant's Critical Philosophy} and in \textit{Essays Critical and Clinical}) Deleuze uses this Shakespearean quote--'the time is out of joint'--to explore Kantian thought. To quote Deleuze: "Hamlet is the first hero who truly needed time to act, whereas earlier heroes were subject to time as the consequence of an original movement (Aeschylus) or an aberrant action (Sophocles). The Critique of Pure Reason is the book of Hamlet, the prince of the north" (ECC, 28).} Deleuze shows how in taking time off its hinges, Kant develops a "sort of modern consciousness of time. " (KST, 1) In this 'modern consciousness' time is separated from the external world of space and thus undergoes a sort of  topological twist. What was once located in the external world is folded in.\footnote{To quote Kant: "Time is not something which exists of itself, or which inheres in things as an objective determination, and it does not, therefore, remain when abstraction is made of all subjective conditions of its intuition\dots Time is nothing but the subjective condition under which alone intuition can take place in us" (CPR, 76).} Time, detached from the movement of that which is outside us becomes the structuring principle which conditions the inside. Thus, as we will see, in making time the form of inner sense Kant not only redefines the classical conception of time, he also transforms the traditional understanding of interiority and its relation to the outside. 

In Plato's dichotomized world, time exists on only one side of the mirror. Essences, which are eternal and real, exist in a transcendent realm outside time. Time, on the other hand, belongs to the world of appearances, which, according to Plato, is governed by a continuous process of change and movement. For Plato, then, it is change and movement that are the defining features of time. 

In his dialogue Timeaus, Plato describes time as the "image of eternity" (T, 1167). He perceives this image in the movement of the stars and thus equates the production of time with the 'perfect and immutable' cycles within which the planets revolve. "Such was the mind and thought of God in the creation of time. The sun and the moon and five other stars which have the name of planets were created by him in order to distinguish and preserve the numbers of time" (T, 1167). In the 'curved time' of a 'circular universe' Plato's God has bent the skies into an arc. "A Demiurge which makes  circles," (SLK, 2) as Deleuze puts it, has created a world whose map is observed in the heavenly spheres. Thus, for Plato, 'the name' and 'the number' of time can be found in the changes and motion that take place in the sky. 

Kant's break with Plato is absolute. For, according to the first Critique, movement is empirical, which is to say it exists at the level of experience. Transcendental philosophy, he writes, "cannot count the concept of alteration among its \textit{a priori} data" (CPR, 75). For Kant the revolutions of the stars, the swing of a pendulum, the sand in an hourglass all occur \textit{in} time and as such fall outside the problematic of critique. Plato's perfect image can not be time, for, "time itself does not alter but only something which is in time" (CPR, 82). According to Kant, time explains the possibility of movement, but movement is not time. "The concept of alteration and with it the concept of motion, as alteration of place," he writes, "is possible only through and in the representation of time" (CPR, 75).

Thus, Kant liberates the form of time from the endless cycle of the Platonic world. Time is no longer contained within the circular revolutions of the planets. Instead, it is the relation of time to motion itself that revolves. In Kant, writes Deleuze, "time is no longer related to the movement which it measures, but movement is related to the time which conditions it". The \textit{Critique of Pure Reason} conceives of a 'straight line' of time that is cut off from its subordination to all that exists in time.\footnote{It is interesting to note that Foucault, when writing of Deleuze, evokes the 'straightening' of time that is initially produced by the first Critique. The circle must be abandoned as a faulty principle of return; we must abandon our tendency to organize everything into a sphere. All things return on the straight and narrow by way of a straight and labyrinth line" (TP, 166). This notion of the straight labyrinth of time is used by Deleuze to designate the transcendental form of time. In the preface to \textit{Kant's Critical Philosophy}, he writes as follows: We move from one labyrinth to another. The labyrinth is no longer a circle, or a spiral which would translate its complications, but a thread, a straight line, all the more mysterious for being simple, inexorable as Borges says, "the labyrinth which is composed of a single straight line, and which is indivisible, incessant" (KCP, vii). Deleuze also uses the notion of a straight labyrinth to describe the time of Aeon (a concept that will be discussed in detail in chapter 3).} To quote from Deleuze:

\begin{quote}
    Time is no longer coiled up in such a way that it is subordinated to the measure of something other than itself, such as, for example, astronomical movement. Everything happens as if, having been coiled up so as to measure the passage of celestial bodies, time unrolls itself like a sort of serpent, it shakes off all subordination to a movement or a nature, it becomes time in itself for itself, it becomes pure and empty time. It measures nothing anymore. Time has taken on its own excessiveness. It is out of its joints, which is to say its subordination to nature; its now nature which is subordinated to it (KST, 12).
\end{quote}

Since time is already presupposed in motion, Kant, 'the great explorer,' must begin to search elsewhere. Looking behind objects to discover the conditions of their production, Kant finds that time is "not an empirical concept that has been derived from any experience" (CPR, 74). In the abstract realm of the transcendental, Kant discovers a form of time that is independent of the experience of motion.

It would first appear that Kant finds a predecessor in Aristotle who modifies the Platonic vision, by making a distinction between motion and time. Though Aristotle connects time to the cycles of astronomical change he does not equate it with them. His argument rests on two points. First, that while time exists everywhere, movement only occurs in particular things. Second, that while things that move can be either fast or slow, time itself does not shift in speed. "It is evident, then," writes Aristotle, "that time is neither movement nor independent of movement" (P, 371).

In order to discover the precise relation between motion and time, Aristotle turns to number. To quote his famous formula "time is just this--the
number of motion" (P, 372). The circular revolutions, the heavenly spheres are still linked to time but the two are no longer directly equivalent. "Time, is only movement," writes Aristotle "in so far as it admits of enumeration" (P, 372). Change must be numbered for it to be time. 

"Time, then," for Aristotle, "is a kind of number" (P, 372). The kind of number that it is, rests on a distinction made in the Physics, between the "number with which we count", and "the number of things which are counted" (P, 373). To quote from the \textit{Physics}: "Number, is used in two ways--both of what is counted or countable and also of that which we count. Time, then, is what is counted, not that with which we count: these are different kinds of thing (P, 372).

Thus, Aristotle, like Kant, seeks to discover time by shifting focus away from the concreteness of astronomy. However, as we will see, the abstraction he makes in the direction of number is ultimately subordinated to empirical movement. For according to Aristotle the numbers of time are determined as "the measure of a quantity of change" (P, 375).

For Kant, as for Aristotle, time is conceived as being fundamentally numeric. Kant recognizes that any representation cannot help but serve to spatialize time. Since time 'yields no shape' (CPR, 77) it cannot be represented. Still, he focuses on a single spatial image that will function as an analogy for his new form of time. For Kant, the closest we get to time in space, is the image of the real number line.\footnote{Commentator Alfredo Ferrarin makes much of this, arguing that in Kant numbering and arithmetic are practically synonymous with the generation of time.} "We represent the time sequence," he writes, "by a line progressing to infinity, in which the manifold constitutes a series of one dimension only; and we reason from the properties of this line to all the properties of time" (CPR 77).\footnote{This sentence continues as follows: "with one exception that while the parts of the line are simultaneous the parts of time are always successive" (CPR, 77). Yet, it is important to note, that while the parts of time are successive, time itself, for Kant, is not. Deleuze is insistent on this point. "Time," he writes, "is no longer defined by succession because succession concerns only things and movements which are in time. If time itself were succession, it would need to succeed in another time, and on to infinity. Things succeed each other in various times, but they are also simultaneous in the same time, and they remain in an indefinite time. It is no longer a question of defining time by succession, nor space by simultaneity, nor permanence by eternity. Permanence, succession and simultaneity are modes and relationships of time" (KCP, vii-iii).}

Yet, despite the fact that both Kant and Aristotle link time to number, Kant breaks with Aristotle no less than with Plato. For in freeing time from its subordination to movement, Kant had to uncouple number from measurement. In the domain of the transcendental, the numerical processes of time are unhinged from the concreteness of change. With Kant the number of time thus breaks from Aristotle's classical formula. Ceasing to function as 'the quantity of motion', the 'things that are counted' become 'that with
which we count'. 

It is easier to understand what is at stake in this reversal if we map Aristotle's distinction in the philosophy of number on to the difference between cardinal and ordinal numbers. Cardinal numbers [one, two, three\dots] are used to express amount or quantity. Ordinal numbers, on the other hand, are used to express position [first, second, third]. Bound to keep track of that which it belongs to, the cardinal number is tied down, attached to what is in time. "Cardinal," writes Deleuze, "comes from cardo; cardo is precisely the hinge, the hinge around which the sphere of celestial bodies turns and which makes them pass time and again through the so-called cardinal points" (KST, 12). Ordinal numbers, on the other hand, are indifferent to the space of measurement.\footnote{A familiar example of ordinal numbering is that used by the library cataloguing system. It is clear, in this case, that what is important is the order of the books and not the amount of space between them.} What counts, for them, is order not measure. In unchaining time from its bonds to what is in time, Kant simultaneously freed number from its subordination to measurement. As Deleuze writes, in Kant's 'new definition of time' number "ceases to be cardinal and becomes ordinal, a pure order of time" (DR, 88).

In the end therefore, despite the differences between Plato's and Aristotle's conceptions of time, from the perspective of transcendental philosophy, they are basically alike. For both belong to "a certain tradition of antiquity, in which time is fundamentally subordinated to something which happens in it, and this something can be determined as being change" (SLK, 1). By separating time from the heavenly spheres and making number independent of motion, Kant thus splits with an entire tradition. Operating with the conjunction 'what appears/conditions of appearance,' Kant creates (or discovers) a disjunction between the form of time itself and the changes which occur in time.

This distinction between the abstract form of time and the changes which occur within it requires that time be released from its spatial determinations. In the \textit{Critique of Pure Reason}, time is no longer located in the objects of the world or in their relations, but is situated instead in an abstract realm that is independent from our perceptions of space. The transcendental form of time, as we have seen, is an empty form, conceived of as nothing but a pure ordinal sequence. Since time has no spatial dimension, even the number line is but an analogy.

In the Transcendental Aesthetic, space, as we have seen, captures the whole of exteriority inside itself. Thus in order to separate time from space, time which was once located in the external world, must be folded in. For, according to the first Critique, the only thing that is not in space is the inner determinations of our mind. In making time the 'form of inner sense' Kant thus locates time in the singular domain which exists outside the representation of space.

This process of interiorization gives time a certain dominance over space. For according to Kant everything that we represent as in space must also be processed by time, precisely insofar as it is experienced (and thus belongs to our inner states). "Appearances," writes Kant, "may, one and all, vanish; but time (as the universal condition of their possibility) cannot itself be removed" (CPR, 75). Thus, for Kant, while everything that exists is in time, time is the one thing that does not exist in space, since everything in space 'already' presupposes time.\footnote{From the perspective of transcendental philosophy, then, William Burroughs' formula, "The only way out of time is into space" (H, 19) can only be a trick.} To quote from the first Critique:

\begin{quote}
    Time is the formal \textit{a priori} condition of all appearances whatsoever. Space, as the pure form of all \textit{outer} intuition, is so far limited; it serves as the \textit{a priori} condition only of outer appearances. But since all representations, whether they have for their objects outer things or not, belong in themselves, as determinations of the mind, to our inner state; and since this inner state stands under the formal condition of inner intuition, and so belongs to time, time is an \textit{a priori} condition of all appearances whatsoever\dots Just as I can say \textit{a priori} that all outer appearances are in space, and are determined \textit{a priori} in conformity with the relations of space, I can also say, from the principle of inner sense, that all appearances whatsoever, that is, all objects of the senses, are in time, and necessarily stand in time relations (CPR, 77).
\end{quote}

Once abstracted and interiorized time takes on enormous new powers. Productive of the actual rhythm of thought and sensation it gains control over the whole of experience. Time, as the form of interiority, is thus absolutely inescapable. For everything we see, think, feel, hear and know has already been given a speed, an order and a rhythm in time. With Kant, then, the inside ceases to be conceived of an empirical container and is instead thought transcendentally (as an interiority over against space and not merely in space). One can never escape time, since time is a limit that works us from the inside. Yet, as Deleuze notes, there is something very strange in this notion of time as interior limit.

\begin{quote}
    To think time means to substitute for the classical schema of an exterior limitation of thought by the extended, the very very strange idea of an interior limit to thought which works it from the inside, which doesn't at all come from outside, which doesn't at all come from the opacity of a substance. As if there was in thought something impossible to think. As if thought was worked over from the inside by something that it cannot think (TLK, 1).
\end{quote}

In moving from transcendence to the transcendental, Kant reworks both the inside and the outside. Time conditions an inescapable interiority, but in doing so opens a new and more radical exteriority, since the production of time itself cannot be captured "within" time. In other words, the one thing that is not interior to time is the transcendental form of time itself. Thus, in discovering the abstract realm of the transcendental, Kant unmasks an unanticipated immanent exteriority--an outside that does not transcend the world but that is no less alien for that. "The greatest initiative of transcendental philosophy," writes Deleuze, "was to introduce the form of time into thought" (DR, 86). Yet, as the next section will show, it is only a very particular mode of thought that can process Kant's modern consciousness of time. For Kant revolutionizes the interiority of thought only by immersing it in the exteriority of time.

\newpage
\section{Transcendental Deduction: Time and the `I think'}

\vskip 4ex
\chapterprecishere{How can man think what he does not think, inhabit by a mute occupation
something that eludes him, animate with a kind of frozen movement that
figure of himself that takes the form of a stubborn exteriority.\par\raggedleft--- \textup{Michel Foucault}, The Order of Things, \textup{p. 323}}

\vskip 4ex
\chapterprecishere{It is false to say: I think. One should say: one thinks me\dots I is another\par\raggedleft--- \textup{Arthur Rimbaud}, Complete Works, \textup{p. 101}}

\vskip 4ex
\chapterprecishere{I am separated from myself by the form of time.\par\raggedleft--- \textup{Gilles Deleuze}, Kant's Critical Philosophy, \textup{p. ix}}



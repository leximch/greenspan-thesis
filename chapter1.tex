\chapter{Philosophy's Copernican Revolution: Time in The \textit{Critique of Pure Reason}}

\chapterprecishere{The history of Immanuel Kant's life is difficult to portray, for he had neither life nor history ... I do not believe that the great clock of the cathedral performed in a more passionless and methodical manner its daily routine than did its townsman, Immanuel Kant. Rising in the morning, coffee drinking, writing, reading lectures, dining, walking, everything had its appointed time, and the neighbours knew that it was exactly half-past three-o-clock when Immanuel Kant stepped forth from his house in his grey, tight fitting coat, with his Spanish cane in his hand, and betook himself to the little linden avenue called after him to this day the `Philosophers walk. "... What a strange contrast did this man's outward life present to his destructive, world annihilating thought! In sooth, had the citizens of Konigsberg had the least presentiment of the full significance of his ideas, they would have felt a far more awful dread at the presence of this man than at the sight of the executioner, who can but kill the body. But the worthy folks saw in him nothing more than a Professor of Philosophy, and as he passed in his customary hour, they greeted him in a friendly manner and set their watches by him\par\raggedleft--- \textup{Heinrich Heine}, Religion and Philosophy in Germany, \textup{p. 109.}}

\newpage
% TODO: Fix quote/epigraph format.
\chapterprecishere{\par\raggedleft\textbf{\textup{49. Ko. Revolution.}} \begin{flushleft}\textbf{\textup{[Molting]}}\end{flushleft}\ Fire in the water. The image of REVOLUTION. Thus the superior man sets the calendar in order and makes the seasons clear.\par\raggedleft---I Ching: The Book of Changes, \textup{p. 190.}}

\section{Introduction: The Discovery of Transcendental Time}

In 1781 Immanuel Kant published the first edition of the \textit{Critique of Pure Reason}. The history of philosophy registers the date as the moment of Kant's "Copernican revolution, " a moment in which Kant is said to have accomplished in the realm of thought what Copernicus had accomplished, over two centuries earlier, in the realm of astronomy. Traditionally, the story of Kant's revolution emphasizes epistemology and concentrates on the role of the human intellect in constituting the external world. It is here, we are told, that one should locate the dramatic shift that is at the core of Kantian philosophy.\footnote{ In the following section this position will be illustrated through the writings of Heinrich
Heine. Though Heine poetic and lyrical language is unique amongst the
commentators on Kant, his views are not at all unconventional. In the '"Past Masters" text on Kant, for example, Roger Scruton, writes that the essence of the Copernican revolution is that 'self consciousness requires that the world must appear to conform to the categories' (Scruton, Roger. \textit{Kant}. Oxford: Oxford University Press, 1982, p. 28). The Encyclopedia of Philosophy concurs that this epistemological view is the key to understanding Kantian thought. 'Kant's principle task in the \textit{Critique of Pure Reason} was to determine the cognitive powers of reason, to find out what it could and could not achieve in the way of knowledge" ("Kant, Immanuel" in the \textit{Encyclopedia of Philosophy Volume 3}. Edited by Paul Edwards. New York: Macmillan Publishing, 1967, p. 305).} What this chapter will argue, however, is that this emphasis, on the role of the intellect, mistakes what is truly revolutionary in critical thought. The argument, which draws on Deleuze's reading of Kant,\footnote{This is comprised both of Deleuze's book on Kant and a series of lectures on Kant that have been reprinted on the Internet (see bibliography).} consists of two main points. First, that the transformation of the human subject is merely a consequence of Kant's more fundamental innovation, the discovery of the realm of the transcendental, and second, that to appreciate the truly revolutionary nature of this discovery we must turn our attention away from the enlightened subject of reason and focus instead on the occulted nature of time.

The most common approach to Kant is to read him as an epistemologist, a philosopher who is predominantly concerned with how knowledge can be justified. "According to this account \textit{The Critique of Pure Reason} centers around the question: "How are \textit{a priori} synthetic judgements possible?" (CPR, 59). To answer this question Kant must begin by defining what he means by the synthetic \textit{a priori}. 

"In the order of time," writes Kant in the preface to the first Critique, "we have no knowledge antecedent to experience, and with experience all our knowledge begins. But though all our knowledge begins with experience it does not follow that it all arises out of experience" (CPR, 41). With this distinction--between knowledge that is based in experience and knowledge that is independent of it--Kant separates the \textit{a priori} from the \textit{a posteriori}. This difference is absolute and rigorously determined. Empirical knowledge which has its sources \textit{a posteriori}, that is, in experience" (CPR-43) gives rise to judgments which are particular and contingent (the sun rose today) while \textit{a priori} knowledge, on the other hand, gives rise to judgments which are universal and necessary (today succeeded yesterday). Since knowledge gained from experience is always particular and contingent it can never be the basis for judgments which are universal and necessary. Thus, Kant writes, "necessity and strict universality are the sure criteria of \textit{a priori} knowledge" (CPR, 44). This difference between the \textit{a priori} and the \textit{a posteriori} is the first basic division which allows Kant to demarcate the singular zone of knowledge that he is concerned with in the first Critique.

Yet this distinction, between knowledge that is based in experience and knowledge that is independent of it, is not in itself sufficient for understanding the central problematic of Kantian thought. In order to discover the transcendental, yet another distinction was required. The two sides of the table had to be split in half. To accomplish this, Kant drew another line on a different axes, cutting across both the \textit{a priori} and the \textit{a posteriori}.\footnote{See Table 1 in the appendix.} This line corresponds to the difference between knowledge that is analytic and knowledge that is synthetic. 

Analytic \textit{a priori} cover logical truths. By drawing out "something that is (covertly) contained within a concept" (CPR, 48) they analyze, elucidate, or explicate what is already implicitly known. Kant's example is as follows: "If I say, for instance 'All bodies are extended' this is an analytic judgment. For do not require to go beyond the concept which I connect with 'body' in order to find extension as bound up with it" (CPR, 48).

Analytic knowledge is restricted to the domain of the a priori. It does not arise empirically. For "judgments of experience, " writes Kant, "are one and all synthetic" (CPR, 49) By the term 'synthetic' Kant is referring to knowledge that 'goes beyond the concept'. To make the judgment 'this body is heavy', for example, is to connect concepts 'synthetically'. For since not all bodies are heavy, one cannot arrive at the concept heavy from an analyses of the concept 'body'.\footnote{On page 49 of the first critique Kant uses this example to explain synthetic judgements which are made from experience.}


% Fix Math handling.
Transcendental philosophy concerns itself with the final box in the table--the one that contains the 'synthetic \textit{a priori}'. The puzzle which it sets for itself is how synthetic knowledge can be produced independently of experience. The prime examples of synthetic a priori knowledge are found within the realm of mathematics. As the commentator Alfredo Ferrarin writes, "If what the Critique shows is the possibility of synthetic \textit{a priori} judgements, it is mathematics that takes advantage of this ampliative principle with greatest confidence and success" (KECI, 147.) '7+5=12', to stick with the example that is repeated throughout the secondary literature on Kant,\footnote{Kant uses this example himself both in the first critique and in the \textit{Prolegomena to Future Metaphysics}.} is a synthetic \textit{a priori} judgment. It is \textit{a priori} since it, like all mathematical propositions, is both universal and necessary. It is synthetic since neither the number 7 nor the number 5 has contained within it the number 12. "The concept of 12," writes Kant, "is by no means already thought in merely thinking the union of 7 and 5; and I may analyze my concept of such a possible sum as long as I please, still I shall never find the 12 in it" (CPR, 53).

Though philosophers had long been concerned with such a priori truths as are found in mathematics it was Kant who first recognized them as being synthetic.\footnote{As Deleuze writes, "analytic a priori judgment that meant something, synthetic a posteriori judgment that meant something, but synthetic a priori judgment, that's truly a monster" (KST, 9).} Thus transcendental philosophy, even when confined to epistemology, is far from being a mere exercise in the catalogue of knowledge. For in questioning the tendency to divide everything between the \textit{a priori} analytic judgment and the \textit{a posteriori} synthetic judgment Kant "exposed the insufficiency of certain philosophical categories and conceptual frames" (KST, 9). Focusing his attention on the synthetic \textit{a priori} he exploded old distinctions and in so doing discovered a powerful new machine. 

According to Kant, synthesis is "a blind but indispensable function of the soul". Though "we are scarcely ever conscious" of its power, without it, he insists, "we should have no knowledge whatsoever" (CPR 112). In focusing on the connections and constructions of this hidden realm, the \textit{Critique of Pure Reason} develops a 'synthesized way of handling philosophy'\footnote{This phrase comes from a question raised by Deleuze in an introductory lecture on Kant. "Why," Deleuze asks, "wouldn't there also be a synthesized or electronic way of handling philosophy" (SLK, 1).} which is not based on analyses of that which is already given but on an "extension of our previously possessed concepts" (CPR, 47). It is, writes Kant "a genuinely new addition to all previous knowledge?" (CPR, 51).

It is this 'new addition' that accounts for the shift in the subject's position which occurs in Kantian thought. What is important to recognize here, is that this shift is a result, a corollary, of the more fundamental discovery of this abstract and productive realm of knowledge. Before Kant, the subject was found buried, submerged underground, chained in the darkness of Plato's cave. According to this traditional vision, the subject was trapped in the body, forced to access the world through the unreliability of the perceptual apparatus. An unfortunate fool, blinded by ignorance, duped into mistaking shadows for reality, the subject could not help but deform the world, mutating it into the falsity of illusion. Philosophy's striving consisted in its promise to provide the escape route. Operating with a truth that depended on a "harmony between the subject and the world," (KST, 5) philosophy struggled to cut the chains, to correct the inherent deformity, to free the prisoner from the world of shadows and illusions. 

Through his discovery of the transcendental, Kant replaces harmony with circuitry. The subject, no longer deceived and defective, becomes productive and constitutive. Having given up the impossible attempt at conforming to the objects of the world, "the rational being discovers he has new power" (KCP, 14). After Kant the objects of the world must conform to us. "The first thing the Copernican Revolution teaches us," writes Deleuze, "is that it is we who are giving the orders" (KCP, 14). The prisoner has become a legislator. 

As previously noted, it is this shift in the subject's position which is traditionally taken to be at the core of Kant's revolutionary thought. In a dramatic passage, the poet Heinrich Heine, describes this, philosophy's Copernican revolution, as follows:

\begin{quote}
    Formerly, when men conceived the world as standing still, and the sun as revolving round it, astronomical calculations failed to agree accurately. But when Copernicus made the sun stand still and the earth revolve round it, behold! Everything accorded admirably. So formerly reason, like the sun moved round the universe of phenomena and sought to throw light upon it. But Kant, bade reason, the sun stand still, and the universe of phenomena now turns round, and is illuminated the moment it comes within the region of the intellectual orb (RPG, 114).
\end{quote}

Yet, while Heine has captured the drama of Kant's discovery, his account reveals a certain problem. For if the stress is on human reason, Kant's allusions to Copernicus are somewhat puzzling. Before Copernicus, Heine reminds us, the earth stood as the central pivot or axis around which everything else revolved. Modem astronomy, which is based on the Copernican system, removed the earth from this central position, making it equal to any other planet. The Copernican revolution thus derailed us from our privileged status in relation to phenomena. Kant, on the other hand, is said to have done the exact opposite. Whereas Copernicus displaced us from the center of the universe, Kant put us there. Why, then, does Kant speak of his philosophy as Copernican? For it would seem that the emphasis on the human intellect is not a sufficient explanation. Perhaps, if we look more closely, we might find some other reason for this seemingly confused analogy. 

In the early years of the 16th century, Nicolaus Copernicus, a polish astronomer, attained immortal fame by overthrowing the Ptolemaic universe of the ancient world. Frustrated with the impossibility of achieving accurate measurements of astronomical movements, Copernicus began to question Ptolemy's geocentric vision. Instead of assuming that the stars revolved around a still earth, Copernicus thought, to quote Kant, "whether he might have better success if he made the spectator to revolve and the stars to remain at rest" (CPR, 23). Copernicus posited a heliocentric world in which the stars no longer measured time. He explained the day by the earth's rotation on its own axes and the year by its annual cycle around the sun. In the Copernican system, then, it is the movement of the earth which marks out the temporality of the astronomical calendar. 

The \textit{Critique of Pure Reason}, writes Kant, proceeds "precisely on the lines of Copernicus' primary hypothesis" (CPR, 23). Inspired by the astronomers' method, Kant attempted an analogous experiment in philosophy. Frustrated by the inherent instability of metaphysics, critical thought seeks to attain more solid foundations by focusing not on the authority of experience but on the conditions which make experience possible. 

It is well known that Copernicus' discovery met with fierce resistance, both from natural philosophy and from the church. For these two institutions were allied in their commitment to maintaining the authority of Aristotle, who had insisted that the earth stood still. This resistance was heightened by the fact that despite Copernicus' findings the world still appeared to conform to Ptolemy's ancient vision. Copernicus was thus responsible for a strange and mysterious revolution in which nothing seemed to change but through which everything has been transformed. It is in this way, as we will see, that Kant is a true Copernican. For the Copernican revolution, whether in astronomy or philosophy, changes nothing at the level of experience. Our perceptions and even the way we talk about those perceptions have not altered. Phenomena remain the same. The sun still appears to revolve. The earth still appears to stand still. External bodies still appear to be in motion. We still say that the sun rises and sets. The difference is, and this is the revolution, that now everybody knows it is only a manner of speaking.

In a series of lectures on the \textit{Critique of Pure Reason}, Gilles Deleuze maps out a singular and original account of Kant's Copernican revolution which is based neither on epistemology nor on a change in the position of the intellect but on a shift in the nature of appearance itself. 

According to Deleuze, the classical tradition structured the world around a basic opposition. "The whole of classical philosophy from Plato onwards," he writes, "seemed to develop itself within the, frame of a duality between sensible appearances and intelligible essences" (KST, 4). Thus, previous to Kant the world was divided between, on the one hand, the degraded realm of sensation which was based on bodily knowledge and experience, and on the other hand, the realm of ideas, pure forms or essences which were transcendent and therefore untainted by the blemishes of sensation. 

For Plato \textit{a priori} knowledge was the proof of transcendence. His dialogues insist that the very fact that there is knowledge independent of experience shows that reason remembers a time when it was unfettered by the body's cage and was free to gaze upon the pure essence of things.\footnote{One of the most famous examples of the Platonic view of the a priori occurs in the \textit{Meno} where Socrates infers the transcendence of the Forms through a slave's knowledge of geometry. See Plato, "Meno" in \textit{Plato--The Collected Dialogues}. edited by Edith Hamilton and Huntington Cairns. Princeton New Jersey: Princeton University Press, 1961 (especially pages 363-374).} The philosophical distinction between \textit{a posteriori} and \textit{a priori} knowledge was thus, for Plato, evidence of the fact that our capture in the illusory realm of phenomena could be opposed to an exteriority characterized by the transcendent truth of the idea.

The \textit{Critique of Pure Reason} overturns the classical tradition by developing a philosophy that is no longer grounded in this basic opposition. "For the disjunctive couple appearance/essence," writes Deleuze, "Kant will substitute the conjunctive couple what appears/conditions of appearance. Everything is new in this" (KST 4).

With Kant, then, phenomena cease to be trapped by the ancient duality. "It's like a bolt of lightening" (KST, 4). The world of appearances vanish. What is left instead, according to Deleuze, is the apparition. "The apparition, is what appears in so far as it appears. Full stop. I don't ask myself if there is something behind, I don't ask myself if it is false or not false. The apparition is not at all captured in the oppositional couple, in the binary distinction where we find a appearances distinct from essences" (KST, 4). 

No longer bound by the fundamental distinction of classical thought, Kant transforms the meaning and implications of \textit{a priori} knowledge. "In the case of the \textit{a priori}, " writes Deleuze, "Kant "borrows a word but he completely renews its sense. " (KST, 1) For unlike Plato, in Kant the \textit{a priori}, as we will see, is associated with the immanence of abstraction and not the transcendence of the eternal forms. To quote from Deleuze: "Kant is the one who discovers the prodigious domain of the transcendental. He is the analogue of the great explorer - not of another world, but of the upper and lower reaches of this one" (DR, 135). In opposition to the transcendent ideas and logic of the analytic \textit{a priori}, the synthetic \textit{a priori} constitute a continuous process of production that is both exterior and immanent to our experience of the world. The basic question of transcendental philosophy "how are \textit{a priori} synthetic judgments possible?" can thus be restated as follows: given a certain experience what are the conditions that went in to producing it? Kant's answer, as we will see, shows that that which is exterior--or independent--of experience is not a transcendent world above us but rather an immanent outside. It is this which he calls the transcendental.\footnote{According to Deleuze, "The whole Kantian notion of the transcendental is created in order to refute the classical notion of the transcendent. The transcendental is above all not the transcendent" (SLK, 7). }

As was noted in the introduction, the classical disjunction between essence and appearance corresponds to the distinction in the philosophy of time between, on the one hand, the phenomena of temporality and change, and, on the other, the essence of eternity. Overturning this classical duality between essence and appearance requires not only a transformation in way we approach phenomena but also a fundamental reinvention in the philosophy of time. As we will see in the chapter which follows, the discovery of the domain of the transcendental ultimately rests on this reinvention--or revolution--in the nature of time. Deleuze, recognizing this, writes in an introduction to Kant that "all the creations and novelties that Kantianism will bring to philosophy turn on a certain problem of time and an entirely new conception of time (KST, 1). It is to this new conception of time which this chapter now turns.

\section{Transcendental Aesthetic: Time as the Form of Inner Sense}
\vskip 4ex
\chapterprecishere{Until now the task we have given ourselves was to represent space, the
moment has come to think time.\par\raggedleft--- \textup{Gilles Deleuze}, Kant: Synthesis and Time, \textup{p. 1.}}
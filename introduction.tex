\chapter*{Introduction}
\addcontentsline{toc}{chapter}{Introduction}

In colloquial language the terms inside and outside are used to demarcate simple spatial relations. The boundaries between them are physical and the passage which connects one to the other, though not always easy to negotiate, is never completely blocked. For no matter how secure, walls can always be scaled, doors opened, and gates unlocked. In philosophical language, however, the terms inside and outside designate a relation that is altogether more impermeable. For whether of an individual subject or organism or of a social code or structure, interiority, as a philosophical concept, indicates an absolute segregation.\footnote{Much contemporary or postmodern thought has dealt with the segregation between inside and out by concentrating on the interiority of language. Deconstruction, in particular, has held that this zone of interiority is so all encompassing that it renders an occupation of the outside impossible (note Derrida's famous phrase: il n'y a pas d'hors du texte). Subversion, from this point of view, can only occur as a disruption from within.} The inside, in this context, is a mode of containment that operates not through physical boundaries but by an imperceptible border which draws the contours of all that can be thought and perceived. It is a contention of this thesis that, when used in this absolute sense, the division between inside and out is not a spatial determination but a temporal one. Existence is an enclosure not because it happens in space but because it locks us in time.\footnote{This claim, that interiority has more to do with time than with space is one of the crucial insights of transcendental philosophy and will be discussed in detail in the chapter 1.} It is perhaps for this reason that one can detect a tendency in both philosophy and religion to oppose the concept of time with notions of liberation, escape and exteriority.\footnote{Though extremely widespread, the notion that liberation is an escape from time is perhaps most clearly expressed in the religions of the East such as Hinduism and Buddhism. Both these religions have developed meditative techniques (yoga) that aim to release the practitioner from the never-ending cycle of time, and both maintain that enlightenment is reached through an escape from the illusion of time, or Maya. The details of this thought and its relation to the arguments which follow are, however, beyond the scope of this thesis.} For an inside that is bounded by temporal rhythms must find its outside in a realm which is exterior to time. 

In the classical western tradition this connection between the philosophy of time and the notion of inside and out is based on a disjunction which opposes time to eternity. This disjunction is articulated most famously by Plato who defines time "as the movable image of eternity," (T, 77) a definition which establishes the interiority of time in opposition to an exteriority which is eternal. Though immensely widespread, this contrast between time and eternity, rests, at least in its classical formulation, as we will see, on a very specific understanding of both the nature of time and of the relation between inside and out.\footnote{In what follows I offer a brief outline of the classical conception of time and eternity. Though this could undoubtedly be criticized for passing over many of the complexities of classical thought, the purpose here is only to sketch the basic contours of this philosophical tradition so as to provide a clearer means of understanding the temporal revolutions which constitute the main focus of this thesis. To this end both Plato's and Aristotle's philosophy of time are discussed in more detail in chapter 1.}

The classical tradition equated time with astronomy. Conceived of as a 'movable image', it was thought to be perceived in the cyclical changes of the heavenly spheres.\footnote{See Plato "Timeaus" in \textit{Plato: The Collected Dialogues}. Ed. Edith Hamilton and Huntington Cairns. Princeton: Princeton University Press, 1961 (especially section 38 -39). Also note the following from \textit{Definitions}: "(\textit{chronos}) time: the motion of the sun, the measure of its course." ("Definitions" in \textit{Plato: Complete Works}. Indianapolis: Hacket Publishing Company, 1997, p 1678.)} Equivalent to celestial movement, time was made the very principle of variation.\footnote{ As we will see, both Plato and Aristotle tie time directly to movement. Though Plotinus argues against this formulation, his notion that time is an activity of the soul shows that he too conforms to the classical tradition in linking time with the variable processes of becoming and change.} Manifesting itself as a never ending process of change and activity, it governed the continuous flow of becoming in which all existence was trapped. 

Thus, it was precisely due to its associations with variation that time
was considered to be a mode of capture. For the classical tradition  considered temporal rhythms, the passage of seasons, and the changes of day into night as an elaborate simulation which belonged only to the created world of phenomena. It held that the processes of movement and change were but a shadow of a timeless realm that lies beyond, and it was inside this world of shadow and mirage that the subject was caught.\footnote{I am referring here, of course, to the writings of Plato. It is interesting to note that in the most famous section of these writings Plato uses a spatial metaphor, the image of an underground cave, to illustrate his conception of interiority. Even here, however, the prisoners are captured not so much by spatial boundaries, but rather by their incapability of breaking free from the entrapment produced by a fascination with the moving images of time. See: Plato. "The Republic." in \textit{Plato: The Collected Dialogues}. Ed. Edith Hamilton and Huntington Cairns. Princeton: Princeton University Press, 1961.} Governed by the variations of matter, entangled in the multiplicity of becoming, humans were held prisoner, duped by the illusory movement of time. 

This vision of time found its direct opposition in the concept of eternity. Differentiated from the temporal image of movement and variation, the eternal was conceived of as a realm of constant stasis. In contrasting change with identity, multiplicity with unity, and becoming with being, it offered an alternative to the world of sensible appearance, and thus constituted a realm which existed outside the phenomenology of time.

It is within this dualistic framework, then, that the exteriority of the eternal must be understood. Eternity, conceived of in opposition to temporality, should not be confused with the ever-lasting which is a continuous extension of time.\footnote{"Because eternity touches each and every time, it is easily confused with the closely related concept of what "always was, is, and will be," or, in a word, the everlasting. But in its own proper concept, the eternal only "is"; only in the present tense can it be said to be or act in any way. Exempted from all having-been and going-to-be, eternity is familiarly defined as timelessness, in distinction from the everlasting (sometimes also called the sempiternal)." \textit{Encyclopedia of Religion: Volume 5}. Edited by Mircea Eliade, p 167.} Situated neither in the deep past nor in the distant future, eternity is not a stretch of time but a timelessness. Co-existing simultaneously with each and every moment, it is the essence of appearances, the constant form which the variations of time can only represent in a shadowy fashion. 

As the essence of time, the eternal was revered as the divine
archetype. Operating from above, it was eternity that created time.
Considered to be the essence of transcendent production, it was able to
create without getting involved in the matter of its creation\footnote{Drawing on the work of such feminist thinkers as Luce Irigaray (esp. \textit{Speculum of the Other Woman}. Trans. Gillian C Gill. Ithaca New York: Cornell University Press, 1985) it appears that the classical distinction between the phenomenon of time and the transcendence of eternity is essentially masculine in nature. For the notion of a transcendent eternity, beyond or above the enclosure of time, only occurs by differentiating the eternal from matter, creation, becoming, and multiplicity--that is from all things traditionally thought of as female. It is interesting to note in this respect that in Hinduism, Shakti, the principle of female power, is sometimes conceived of as time. See Zimmer, Heimrich. \textit{Myths and Symbols in Indian Art and Civilization}. Edited by Joseph Campbell. New York: Pantheon Books, 1963.} "Eternity, " writes Plotinus "is a majestic thing and thought declares it identical with the God." (TE, 226)\footnote{The influence of Neoplatonism on monotheistic religion, and in particular on Christianity is much too vast a subject to address here. Suffice it to note that it is clear--even from this minimal exposition--that the philosophical understanding of the eternal converges with the Judeo-Christian God.}

With this quote the obvious convergence between the classical notion
of eternity and the conception of the deity produced by the monotheistic
traditions of the West is clearly revealed. The form of the eternal--as the
genesis of the image of time--is paralleled in the opening section of the Old Testament where we encounter God, as the eternal, who is presented as the creator of time. 

\begin{quote}
    And God said, Let there be light: and there was light. And God saw the light, that it was good: and God divided the light from the darkness. And God called the light day, and the darkness he called night. And the evening and the morning were the first day. (Genesis, 1: 3- 4) 
\end{quote}

According to the bible, then, time originates with the first act of creation. Moving across the waters of the unformed void, God acts initially to generate light and begin the passage of time. This primordial event is the singular occurrence which takes place outside the confines of temporality. For, once it is established, a time determined by the passage of day into night structures and conditions the rest of the week of creation. Eternity, as an exterior and transcendent dimension, appears in its pure form only once again, at the end of the bible, with the promise of messianic redemption.

Thus, in both classical philosophy and in the scriptures of the
Abrahamic traditions, life's entrapment in the continuous process of temporal change, becoming and multiplicity was opposed to the wholeness and unity of an eternal being who, untainted by the material world, existed as complete perfection contained within itself.\footnote{Note the following quote from Plotinus: "That which neither has been nor will be, but simply possesses being; that which enjoys stable existence as neither in process of change nor having ever changed--that is Eternity. Thus we come to the definition: the Life--instantaneously entire, complete, at no point broken into period or part--which belongs to Authentic Existent by its very existence, this is the thing we are probing for--this is Eternity. (TE, 225) } The outside was thus equated with transcendence, a mode of escape that led out of the enclosure of time and allowed one to reach--whether through faith or through knowledge--a higher and more primary inside.

This thesis is an exploration of two revolutions, one philosophical, the other socio-economic which, together, have fundamentally altered the philosophy, culture and technics of time.\footnote{In order to achieve focus, much that relates to the main theme of this thesis--that is, revolutions in the nature of time--was of necessity left out. The two most obvious exclusions are, on the 'materialist side,' the recent changes in the physics of time (a topic that is introduced in an interesting and comprehensive manner by Ilya Prigogene and Isabelle Stengers in \textit{Order out of Chaos}. Bolder Co: New Science Library, 1984) and, on the 'philosophical side', the work of Henri Bergson. (who is undoubtedly an important influence on the philosophy of time found in the writings of Deleuze and Guattari--see Deleuze, Gilles. \textit{Bergsonism}. Trans. Hugh Tomlinson and Barabara Habberjam. New York: Zone Books, 1988). To incorporate these topics would require much more time and space than is available here.} The first of these is philosophy's 'Copernican revolution' which was instigated by Immanuel Kant in his text the Critique of Pure Reason. The second occurs with the onset of capitalism, and involves the invention--and subsequent innovations--of a time-keeping  system that is based on the clock. By bringing these two revolutions together, the thesis seeks to establish a connection between abstract conceptual thought and concrete material practices, a connection which is exemplified by the convergence between the transcendental philosophy of time and the socio-history of time-keeping practices. Establishing this connection, however, requires not only a reformulation of the classical conception of time--produced, as we will see, through the creation of a split between, on the one hand, the constant structure of formal time, and, on the other, empirical change conceived of as history--but also a reinvention of the classical notion of eternity. This latter is found in the work of Deleuze and Guattari who substitute the transcendence of eternity with the immanent concept of Aeon, or the absolute Outside, conceived of as the continuous variation of an intensive temporality. It is by way of this concept of Aeon that we will find, in what appears as the "history" of capitalist time, Aeonic events which are at once entirely abstract and fully material. The abstract materiality of these events, as we will see, transfigure the boundaries between inside and out, for though they are in no way eternal they nevertheless occur on an exterior plane outside the interior confines of time. 

From the point of view of the philosophy of time, the revolutionary break brought on by both Kant and capitalism rests on a transformation which occurs in how time is mapped on to the distinction between constant and variable.\footnote{This distinction corresponds to a set of oppositions including quantity and quality, and content and expression which will be discussed throughout this thesis. It is a contention of the thesis that these oppositional couples--or stratified distinctions--are what constitutes the interiority of time.} As we will see in the chapters which follow, both critique and clocktime differentiate themselves from the classical tradition by insisting that it is not time itself which varies, but rather that variation inheres in that which exists in time. This distinction, between time and that which is in time, arises from the fact that both Kant and capitalism separate temporality from the changing patterns of astronomical cycles. Split off from the concrete rhythms of the phenomenal world, time becomes an abstract grid, the \textit{a priori} frame which structures both philosophical thought and the socio-economic and cultural milieu. Time is no longer variable since it has become the very presupposition of change. The stasis of eternity is thus replaced by the constant fixture of formal time.

This transformation in the nature of time is, as will be made clear, of fundamental importance to the whole of the Kantian system. For it is this division, between time and what occurs in time, which ultimately distinguishes the empirical (a posteriori) from the transcendental (a priori). Kant first insists that time cannot be equated with alteration in the 'Transcendental Aesthetic', the very first section of the Critique. "Alteration is an empirical phenomenon," claims Kant, and is thus "only possible through and in the representation of time." (CPR, 76) This paves the way for what Deleuze has called the 'first great Kantian reversal' (KCP, vii) which frees time from its age old subordination to movement. Unhinged from its ties to change and activity, time becomes an abstract condition of experience, the \textit{a priori} structure within which all change and movement takes place.

In capitalism this differentiation--between a constant temporality and the variation of that which occurs in time--receives concrete expression through the division between clocks and calendars. Though this split has existed for thousands of years,\footnote{Sundials have been in use since the third millennium B. C. and evidence of water clocks (or clepsydra) have been found as early as the sixth century B.C. (HH, 20-21).} it is only within capitalism that the distinction between these two types of time-keeping devices has become an abstract distinction in the nature of time itself. Through the continuous innovation and growing ubiquity of the clock, capitalism contrasts the qualitative time of the calendar (with its differences in seasons, light, temperature etc) with the precise, homogenous, standardized and purely quantitative ticking of the clock. It is within the formers qualitative time that variation takes place. Change in time is recorded by the calendar which has ceased to measure the rhythms of everyday life and become instead a mechanism subordinated to the developmental narrative of history. Though capitalism makes use of these  variations in calendric time,\footnote{This is a crucial point that will be made clear in our discussion of the economist Boehm-Bawerk found in chapter 2.} it is also essential for the capitalist mode of production that time be treated as an abstract quantity that does not vary. It is this that is provided by the time of the clock. 

By establishing the difference between the structure of a constant temporality and the variable experiences of history, both Kant and capitalism have created a fracture in the appearance of time. It is by way of this fracture, (and its coinciding synthesis) that these two revolutions have managed to overturn the classical tradition and inaugurate what may be called the modern conception of time.\footnote{Deleuze uses this phrase when describing the Kantian conception of time. Though I have not included any strict definition of modernity, I have used the term to describe both the Kantian philosophy of time and the time-keeping practices that developed with the clock. Used In this manner it is meant to differentiate both Kant and clock time from, on the one hand, the philosophy of time upheld in the classical tradition, and, on the other, from the contemporary or 'postmodern' time-keeping practices that have emerged within cyberspace.}

Yet, despite the fact that the critical understanding of temporality finds its parallel in the culture and technics of capitalism there is an adamant insistence, on both sides, that a fundamental distinction be maintained between the philosophy of time and its socio-economic and cultural manifestations. This distinction rests, as we will see, on the apparent divergence between transcendental and historical production. The explicit aim of The Critique of Pure Reason is to establish, through immanence of criteria, the legitimate domain of reason and thereby dismiss the 'groundless pretensions' of metaphysical speculation.\footnote{To quote Kant's famous passage: "it is obviously the effect not of levity but of the matured judgment of the age, which refuses to be put off with illusory knowledge. It is a call to reason to undertake anew the most difficult of all its tasks, namely that of self-knowledge, and to institute a tribunal which will assure to reason its lawful claims, and dismiss all groundless pretensions, not by despotic decrees, but in accordance with its own eternal and unalterable laws. This tribunal is no other than the \textit{critique of pure reason}." (CPR, 9)} From the point of view of critique, therefore, it is strictly illegitimate to hold that time is the product (or the image) of eternity. In revolt against this classical doctrine, Kant replaces transcendent creation with the immanent synthesis of the understanding. Operating in a realm which is constitutive of experience these synthetic processes construct time as an \textit{a priori} epistemological representation. This representation, which Kant calls the form of inner sense, is the universal and necessary precondition for all empirical phenomena. Put simply then, time, for Kant, is a mental construct within which empirical reality takes place. History, which develops in time, cannot be equated with transcendental synthesis since the very existence of history presupposes and is dependent upon the transcendental construction of time. 

Karl Marx, the most famous philosopher of capitalism, shares Kant's insistence of the need to develop an account of production which does not seek recourse in divine transcendence.\footnote{This is obvious from the famous Marxist contention that religion is the opiate of the people.} However, unlike Kant, Marx maintains that the ultimate realm of production lies not in the synthetic processes of reason but rather in the dialectical forces of history. Thus, for Marx, the \textit{a priori} are themselves subject to change. Produced by the dynamic forces of history, formal time is not an epistemological representation but a contingent historical formation. Marx's historical materialism,\footnote{"According to Engels' 1892 introduction to Socialism: Utopian and Scientific, historical materialism designate[s] that view of the course of history which seeks the ultimate cause and the great moving power of all important historical events in the economic development of society, in the changes in the modes of production and exchange, in the consequent division of society into distinct classes, and in the struggle of these classes against one another." (DMT, 234)} (his "Hegelianism turned on its head"), thus maintains that outside the particularities of the capitalist time machine is a form of variable time with a logic of its own. The exteriority of this temporality - which is not exhaustively structured by any specific mode of production - is ultimately responsible for creating the time of capitalism (conceived of as both the duration of the capitalist mode of production and the structure of time prevailing within it).

Thus, both critical thought which refuses to acknowledge its socioeconomic surroundings, and Marxism which denies the possibility of transcendental synthesis, insist that--despite their obvious connections--the philosophical and socio-technical revolution of time be kept separate and opposed. This opposition, as we have seen, ultimately rests on the fact that the privilege given by transcendental production to the ahistoricity of a constant time comes into conflict with the primacy that historical materialism grants to the variations of a temporality governed by the logic of events. With this conflict, the path to exteriority--on both sides--is lost, as each revolution seeks to contain the other by presenting itself as a higher and more primary inside. Neither Kantian thought, nor the Marxist analyses of capitalism, then, will accept that the exterior realm productive of time is constituted by the eternal transcendence of God. Yet; they nevertheless come into conflict over what should substitute for eternal creation in the modern conception of time. Transcendental critique, the critique of political economy and the secular time of capitalist societies thus converge in their understanding of time but diverge in their accounts of what lies outside it as the ultimate force of production. The modem revolution in the nature of time is thus only partially complete. For though the classical conception of time has been overturned, the notion of eternity--the traditional zone exterior to time--has been left basically unchanged (if only by being ignored). 

Deleuze and Guattari's \textit{Capitalism and Schizophrenia} presents itself as a revolution in transcendental thought which seeks to replace Kantian idealism with a type of Spinozistic materialism.\footnote{Deleuze and Guattari's involvement with Spinozisic philosophy is hard to overestimate. Spinoza's nonreductive immanent, cosmic and ethical materialism could be said to be the single most important influence in their work. Though the philosophy of Spinoza will be touched upon in chapter 3, to explore it in detail is beyond the scope of this thesis. To fully engage with this topic see Deleuze's two books on Spinoza (both listed in the bibliography) and the numerous references to Spinoza found in \textit{A Thousand Plateaus}.} This involves, as we will see, a critique of the Kantian system itself. For according to Deleuze and Guattari, the Kantian notion that transcendental production occurs under the unity of the subject and is therefore epistemological in nature, is strictly illegitimate from the viewpoint of critique. Refusing to see a priori synthesis as an idealist representation, they reconstruct transcendental philosophy on the basis of an immanent materialism. This combines the critical method with a Spinozistic vision of a world laid out on a single plane (substance or Nature).\footnote{ To quote from Deleuze: "Everyone knows the first principle of Spinoza: one substance for all the attributes. But we also know the third, fourth or fifth principle: one Nature for all bodies, one Nature for all individuals, a Nature that is itself an individual varying in an infinite number of ways. What is involved is no longer the affirmation of a single substance, but rather the laying out of a \textit{common plane of immanence} on which all bodies, all minds and all individuals are situated." (SPP, 122)} Transcendental synthesis, thus cease to function as the interior operations of reason and become instead machinic\footnote{The term machine will be used throughout the thesis as it is crucial to the work of Deleuze and Guattari. It will be explained in more detail in chapter 3. Briefly, though, Deleuze and Guattari use the term machine, not to signify a technical apparatus, but rather to designate the immanent circuits of production that constitute any flat assemblage (regardless of its particular form or substance).} diagrams for the intensive multiplicities that compose and populate an exterior body, which Deleuze and Guattari call the plane of consistency, planomenon, or body without organs.

Whereas Kant's Copernican revolution involved a reformulation of the nature of time, so \textit{Capitalism and Schizophrenia}'s materialism involves a revolution in the nature of eternity. This requires, as we will see, that the opposition between an interiorized notion of time, associated with change, multiplicity and becoming, and the conception of the outside as a divine, transcendent and unified eternity, be overturned. Transcendental materialism thus substitutes the classical disjunction between time and eternity with the difference between two planes of composition which function machinically to produce the distinction between extensive and intensive time. The former of these--named Chronos--is attributed to the plane of organization and development, while the latter belongs to the immanent plane of consistency and is given the name of Aeon. With the concept of Aeon, Deleuze and Guattari bring to philosophy a notion of eternity which is not based on the wholeness and unity of a transcendent beyond but on the flat multiplicity of an immanent outside.

In the biblical tradition the eternal cuts into time through singular events that are explosive and highly dramatic in nature. At the limit, it appears as genesis and apocalypse, the beginning and end of creation. Beyond these points, the eternal is encountered only after death, on judgment day, where it carries the threat or promise of damnation and salvation, or when it crashes into history interrupting the linear order of time through miracles and divine revelation. 

We will see that--though no less intense--the connection between Aeon and Chronos is much more quiet and subtle. For Aeon does not manifest itself in time. Though it is itself composed of singular events--which can be precisely dated and named--these events compose a virtual plane of intensity that positively avoids climactic actualization. Deleuze and Guattari call these Aeonic occurrences plateaus and show how they constitute an exteriority that haunts the successive order of extensive temporality.

The final chapter of the thesis takes the pervasive sense of anticlimax that accompanied the dawn of the third millennium as indexing one such event and explores Y2K--a sign that operates as both a date and a name--as a singular Aeonic occurrence. While this may first appear farfetched, we will see that, though it has now been dismissed as irrelevant, Y2K is crucial to the transcendental philosophy of time. This is primarily due to the fact that, as a singularity, it shares all the characteristic features of Aeon including: an effective virtuality, a nonsignifying semiotic, a disruption--or positive avoidance--of extensive succession and an immanent machinic abstraction.

Cutting across the stratified segmentation of Chronos, Y2K thus functions as a mutation (or accident) both in the structure of formal time and in the empirical development of history. It collapses the distinction between time's formal expression and the content which happen to fill it, dissolving the rigid opposition between technics and culture, constant and variable and temporality and change. In this way Y2K constitutes an event--not in time but of time--that allows the capitalist production of temporality to escape from the interiority of history and thus exemplifies the convergence between the material practices of time-keeping systems and processes of abstraction which are conventionally located in the philosophy of time.